\documentclass{hotnets22}

\usepackage{times}  
\usepackage{hyperref}
\usepackage{titlesec}

\hypersetup{pdfstartview=FitH,pdfpagelayout=SinglePage}

\setlength\paperheight {11in}
\setlength\paperwidth {8.5in}
\setlength{\textwidth}{7in}
\setlength{\textheight}{9.25in}
\setlength{\oddsidemargin}{-.25in}
\setlength{\evensidemargin}{-.25in}

\begin{document}

% \conferenceinfo{HotNets 2022} {}
% \CopyrightYear{2022}
% \crdata{X}
% \date{}

%%%%%%%%%%%% THIS IS WHERE WE PUT IN THE TITLE AND AUTHORS %%%%%%%%%%%%

\title{HotNets 2022 Paper}

\author{Paper \#0, 3 pages}

\maketitle

%%%%%%%%%%%%%  ABSTRACT GOES HERE %%%%%%%%%%%%%%
\begin{abstract}

Everybody loves TCP~\cite{vanjacobson}. The paper body in this example
using the HotNets 2022 style file contains a copy
of the CFP text to show the correct format of a standard text page.

\end{abstract}

\section{Call for Papers}

The 21st ACM Workshop on Hot Topics in Networks (HotNets 2022) will bring together researchers in computer networks and systems to engage in a lively debate on the theory and practice of computer networking. HotNets provides a venue for discussing innovative ideas and for debating future research agendas in networking.

We invite researchers and practitioners to submit short position papers. We encourage papers that identify fundamental open questions, advocate a new approach, offer a constructive critique of the state of networking research, re-frame or debunk existing work, report unexpected early results from a deployment, report on promising but unproven ideas, or propose new evaluation methods. Novel ideas need not be supported by full evaluations; well-reasoned arguments or preliminary evaluations can support the possibility of the paper's claims.

We seek early-stage work, where the authors can benefit from community feedback. An ideal submission has the potential to open a line of inquiry for the community that results in multiple conference papers in related venues (SIGCOMM, NSDI, CoNEXT, SOSP, OSDI, MobiCom, MobiSys, etc.), rather than a single follow-on conference paper. The program committee will explicitly favor early work and papers likely to stimulate reflection and discussion over ``conference papers in miniature''. Finished work that fits in a short paper is likely a better fit with the short-paper tracks at either CoNEXT or IMC, for example.

HotNets takes a broad view of networking research. This includes new ideas relating to (but not limited to) mobile, wide-area, data-center, home, and enterprise networks using a variety of link technologies (wired, wireless, visual, and acoustic), as well as social networks and network architecture. It encompasses all aspects of networks, including (but not limited to) packet-processing hardware and software, virtualization, mobility, provisioning and resource management, performance, energy consumption, topology, robustness and security, measurement, diagnosis, verification, privacy, economics and evolution, interactions with applications, and usability of underlying networking technologies.

Position papers will be selected based on originality, likelihood of stimulating insightful discussion at the workshop, and technical merit. Accepted papers will be posted online prior to the workshop and will be published in the ACM Digital Library to facilitate wide dissemination of the ideas discussed at the workshop.


\section{Concurrent Submission Policy}

Concurrent submissions to HotNets 2022 and any other peer-reviewed venue that cover the same work (differences in degree of detail given the two venues' length limits notwithstanding) are prohibited, and will result in the immediate rejection of the HotNets submission in question. ``Concurrent'' means any other peer-reviewed venue whose reviewing period (i.e., between submission and notification) overlaps with that of HotNets. The ``same work'' means, for example, a submission overlapping significantly in content with a conference submission. However, a position paper submitted to HotNets (e.g., that reflects broadly on the state of some aspect of the field, adopts a position as to how the field should move forward, or articulates a broad avenue of future work) will not be considered the ``same work'' as a conference-length paper on a specific system that addresses a point under the broad umbrella covered by the position paper. Authors with questions about the concurrent submission policy are encouraged to contact the PC chairs (Radhika and Ranveer) prior to submitting.

\section{Ethical concerns}

As part of the submission process, authors must attest that their work complies with all applicable ethical standards of their home institution(s), including, but not limited to privacy policies and policies on experiments involving humans. Note that submitting research for approval by one's institution's ethics review body is necessary, but not sufficient---in cases where the PC has concerns about the ethics of the work in a submission, the PC will have its own discussion of the ethics of that work. The PC takes a broad view of what constitutes an ethical concern, and authors agree to be available at any time during the review process to rapidly respond to queries from the PC chairs regarding ethical standards.

\section{Submission and Formatting}

Submitted papers must be no longer than 6 pages (10 point font, 12 point leading, 7 inch by 9.25 inch text block) including all content except references. There are no limits for reference pages[1]. This tarball contains a LaTeX class file that follows the prescribed submission format and an example of a paper formatted using that class.

All submissions must be blind: they must not indicate the names or affiliations of the authors in the paper. Only electronic submissions in PDF will be accepted. Submissions must be written in English, render without error using standard tools (e.g., Acrobat Reader), and print on US Letter paper. Papers must contain novel ideas and must differ significantly in content from previously published papers.

\section*{Acknowledgments}

Lorem ipsum dolor sit amet, consectetur adipisicing elit, sed do eiusmod tempor incididunt ut labore et dolore magna aliqua. Ut enim ad minim veniam, quis nostrud exercitation ullamco laboris nisi ut aliquip ex ea commodo consequat. Duis aute irure dolor in reprehenderit in voluptate velit esse cillum dolore eu fugiat nulla pariatur. Excepteur sint occaecat cupidatat non proident, sunt in culpa qui officia deserunt mollit anim id est laborum.

\bibliographystyle{abbrv} 
\begin{small}
\bibliography{hotnets22}
\end{small}

\end{document}

